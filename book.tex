\PassOptionsToPackage{unicode=true}{hyperref} % options for packages loaded elsewhere
\PassOptionsToPackage{hyphens}{url}
\PassOptionsToPackage{dvipsnames,svgnames*,x11names*}{xcolor}
%
\documentclass[a4paper,]{report}
\usepackage{lmodern}
\usepackage{amssymb,amsmath}
\usepackage{ifxetex,ifluatex}
\usepackage{fixltx2e} % provides \textsubscript
\ifnum 0\ifxetex 1\fi\ifluatex 1\fi=0 % if pdftex
  \usepackage[T1]{fontenc}
  \usepackage[utf8]{inputenc}
  \usepackage{textcomp} % provides euro and other symbols
\else % if luatex or xelatex
  \usepackage{unicode-math}
  \defaultfontfeatures{Ligatures=TeX,Scale=MatchLowercase}
\fi
% use upquote if available, for straight quotes in verbatim environments
\IfFileExists{upquote.sty}{\usepackage{upquote}}{}
% use microtype if available
\IfFileExists{microtype.sty}{%
\usepackage[]{microtype}
\UseMicrotypeSet[protrusion]{basicmath} % disable protrusion for tt fonts
}{}
\IfFileExists{parskip.sty}{%
\usepackage{parskip}
}{% else
\setlength{\parindent}{0pt}
\setlength{\parskip}{6pt plus 2pt minus 1pt}
}
\usepackage{xcolor}
\usepackage{hyperref}
\hypersetup{
            pdftitle={Low-Level Software Security for Compiler Developers},
            colorlinks=true,
            linkcolor=Maroon,
            filecolor=Maroon,
            citecolor=Blue,
            urlcolor=Blue,
            breaklinks=true}
\urlstyle{same}  % don't use monospace font for urls
\setlength{\emergencystretch}{3em}  % prevent overfull lines
\providecommand{\tightlist}{%
  \setlength{\itemsep}{0pt}\setlength{\parskip}{0pt}}
\setcounter{secnumdepth}{5}
% Redefines (sub)paragraphs to behave more like sections
\ifx\paragraph\undefined\else
\let\oldparagraph\paragraph
\renewcommand{\paragraph}[1]{\oldparagraph{#1}\mbox{}}
\fi
\ifx\subparagraph\undefined\else
\let\oldsubparagraph\subparagraph
\renewcommand{\subparagraph}[1]{\oldsubparagraph{#1}\mbox{}}
\fi

% set default figure placement to htbp
\makeatletter
\def\fps@figure{htbp}
\makeatother

\usepackage{makeidx}
\makeindex
\newcounter{TodoCounter}
\usepackage[backgroundcolor=white,linecolor=black]{todonotes}
\let\oldtodo\todo
\usepackage{bclogo}%  \bcpanchant
\renewcommand{\todo}[1]{
  \stepcounter{TodoCounter}
  \oldtodo[caption={\arabic{TodoCounter}. #1}]{\bcpanchant #1}
}
\newcommand{\missingcontent}[1]{
  \stepcounter{TodoCounter}
  \oldtodo[inline,caption={\arabic{TodoCounter}. #1}]{\bcpanchant \textit{#1}}
}

\title{Low-Level Software Security for Compiler Developers}
\date{}

\begin{document}
\maketitle

\clearpage

\vspace*{\fill}
%\includegraphics[height=2cm]{example-image}
This work is licensed under the Creative Commons Attribution 4.0 International
License. To view a copy of this license, visit
http://creativecommons.org/licenses/by/4.0/ or send a letter to Creative
Commons, PO Box 1866, Mountain View, CA 94042, USA.

  © 2021 Arm Limited
  \href{mailto:kristof.beyls@arm.com}{\nolinkurl{kristof.beyls@arm.com}}\\

Version 0-74-g85fd1a2
\clearpage

{
\hypersetup{linkcolor=}
\setcounter{tocdepth}{2}
\tableofcontents
}
\hypertarget{introduction}{%
\chapter{Introduction}\label{introduction}}

Compilers, assemblers and similar tools generate all the binary code
that processors execute. It is no surprise then that for security
analysis and hardening relevant for binary code, these tools have a
major role to play. Often the only practical way to protect all binaries
with a particular security hardening method is to let the compiler adapt
its automatic code generation.

With software security becoming even more important in recent years, it
is no surprise to see an ever increasing variety of security hardening
features and mitigations against vulnerabilities implemented in
compilers.

Indeed, compared to a few decades ago, today's compiler developer is
much more likely to work on security features, at least some of their
time.

Furthermore, with the ever-expanding range of techniques implemented, it
has become very hard to gain a basic understanding of all security
features implemented in typical compilers.

This poses a practical problem: compiler developers must be able to work
on security hardening features, yet it is hard to gain a good basic
understanding of such compiler features.

This book aims to help developers of code generation tools such as JITs,
compilers, linkers and assemblers to overcome this.

There is a lot of material that can be found explaining individual
vulnerabilities or attack vectors. There are also lots of presentations
explaining specific exploits. But there seems to be a limited set of
material that gives a structured overview of all vulnerabilities and
exploits for which a code generator could play a role in protecting
against them.

This book aims to provide such a structured, broad overview. It does not
necessarily go into full details. Instead it aims to give a thorough
description of all relevant high-level aspects of attacks,
vulnerabilities, mitigations and hardening techniques. For further
details, this book provides pointers to material with more details on
specific techniques.

The purpose of this book is to serve as a guide to every compiler
developer that needs to learn about software security relevant to
compilers. Even though the focus is on compiler developers, we expect
that this book will also be useful to other people working on low-level
software.

\hypertarget{why-an-open-source-book}{%
\section{Why an open source book?}\label{why-an-open-source-book}}

The idea for this book emerged out of a frustration of not finding a
good overview on this topic. Kristof Beyls and Georgia Kouveli, both
compiler engineers working on security features, wished a book like this
would exist. After not finding such a book, they decided to try and
write one themselves. They immediately realized that they do not have
all necessary expertise themselves to complete such a daunting task. So
they decided to try and create this book in an open source style,
seeking contributions from many experts.

As you read this, the book remains unfinished. This book may well never
be finished, as new vulnerabilities continue to be discovered regularly.
Our hope is that developing the book as an open source project will
allow for it to continue to evolve and improve. The open source
development process of this book increases the likelihood that it
remains relevant as new vulnerabilities and mitigations emerge.

Kristof and Georgia, the initial authors, are far from experts on all
possible vulnerabilities. So what is the plan to get high quality
content to cover all relevant topics? It is two-fold.

First, by studying specific topics, they hope to gain enough knowledge
to write up a good summary for this book.

Second, they very much invite and welcome contributions. If you're
interested in potentially contributing content, please go to the home
location for the open source project at
\url{https://github.com/llsoftsec/llsoftsecbook}.

As a reader, you can also contribute to making this book better. We
highly encourage feedback, both positive and constructive criticisms. We
prefer feedback to be received through
\url{https://github.com/llsoftsec/llsoftsecbook}.

\missingcontent{Add section describing the structure of the rest of the book.}

\hypertarget{memory-vulnerability-based-attacks-and-mitigations}{%
\chapter{Memory vulnerability based attacks and
mitigations}\label{memory-vulnerability-based-attacks-and-mitigations}}

\hypertarget{a-bit-of-background-on-memory-vulnerabilities}{%
\section{A bit of background on memory
vulnerabilities}\label{a-bit-of-background-on-memory-vulnerabilities}}

Memory access errors describe memory accesses that, although permitted
by a program, were not intended by the programmer. These types of errors
are usually defined (Hicks \protect\hyperlink{ref-Hicks2014}{2014}) by
explicitly listing their types, which include:

\begin{itemize}
\tightlist
\item
  buffer overflow
\item
  null pointer dereference
\item
  use after free
\item
  use of uninitialized memory
\item
  illegal free
\end{itemize}

Memory vulnerabilities are an important class of vulnerabilities that
arise due to these types of errors, and they most commonly occur due to
programming mistakes when using languages such as C/C++. These languages
do not provide mechanisms to protect against memory access errors by
default. An attacker can exploit such vulnerabilities to leak sensitive
data or overwrite critical memory locations and gain control of the
vulnerable program.

Memory vulnerabilities have a long history. The
\href{https://en.wikipedia.org/wiki/Morris_worm}{Morris worm} in 1988
was the first widely publicized attack exploiting a buffer overflow.
Later, in the mid-90s, a few famous write-ups describing buffer
overflows appeared (Aleph One
\protect\hyperlink{ref-AlephOne1996}{1996}).
\protect\hyperlink{stack-buffer-overflows}{Stack buffer overflows} were
mitigated with \protect\hyperlink{stack-buffer-overflows}{stack
canaries} and \protect\hyperlink{stack-buffer-overflows}{non-executable
stacks}. The answer was more ingenious ways to bypass these mitigations:
\protect\hyperlink{code-reuse-attacks}{code reuse attacks}, starting
with attacks like
\protect\hyperlink{code-reuse-attacks}{return-into-libc} (Solar Designer
\protect\hyperlink{ref-Solar1997}{1997}). Code reuse attacks later
evolved to \protect\hyperlink{code-reuse-attacks}{Return-Oriented
Programming (ROP)} (Shacham \protect\hyperlink{ref-Shacham2007}{2007})
and even more complex techniques.

To defend against code reuse attacks, the
\protect\hyperlink{code-reuse-attacks}{Address Space Layout
Randomization (ASLR)} and
\protect\hyperlink{code-reuse-attacks}{Control-Flow Integrity (CFI)}
measures were introduced. \todo{Refine section
links used here and in the previous paragraph.} This interaction between
offensive and defensive security research has been essential to
improving security, and continues to this day. Each newly deployed
mitigation results in attempts, often successful, to bypass it, or in
alternative, more complex exploitation techniques, and even tools to
automate them.

Memory safe (Hicks \protect\hyperlink{ref-Hicks2014}{2014}) languages
are designed with prevention of such vulnerabilities in mind and use
techniques such as bounds checking and automatic memory management. If
these languages promise to eliminate memory vulnerabilities, why are we
still discussing this topic?

On the one hand, C and C++ remain very popular languages, particular in
the implementation of low-level software. On the other hand, programs
written in memory safe languages can themselves be vulnerable to memory
errors as a result of bugs in how they are implemented, e.g.~a bug in
their compiler. Can we fix the problem by also using memory safe
languages for the compiler and runtime implementation? Even if that were
as simple as it sounds, unfortunately there are types of programming
errors that these languages cannot protect against. For example, a
logical error in the implementation of a compiler or runtime for a
memory safe language can lead to a memory access error not being
detected. We will see examples of such logic errors in compiler
optimizations in a
\protect\hyperlink{jit-compiler-vulnerabilities}{later section}.

Given the rich history of memory vulnerabilities and mitigations and the
active developments in this area, compiler developers are likely to
encounter some of these issues over the course of their careers. This
chapter aims to serve as an introduction to this area. We start with a
discussion of exploitation primitives, which can be useful when
analyzing threat models \todo{Discuss
threat models elsewhere in book and refer to that section here}. We then
continue with a more detailed discussion of the various types of
vulnerabilities, along with their mitigations, presented in a rough
chronological order of their appearance, and, therefore, complexity.

\hypertarget{exploitation-primitives}{%
\section{Exploitation primitives}\label{exploitation-primitives}}

Newcomers to the area of software security may find themselves lost in
many blog posts and other publications describing specific memory
vulnerabilities and how to exploit them. Two very common, yet unfamiliar
to a newcomer, terms that appear in such publications are \emph{read
primitive} and \emph{write primitive}. In order to understand memory
vulnerabilities and be able to design effective mitigations, it's
important to understand what these terms mean, how these primitives
could be obtained by an attacker, and how they can be used.

An \emph{exploit primitive}\index{exploit primitive} is a mechanism that
allows an attacker to perform a specific operation in the memory space
of the victim program. This is done by providing specially crafted input
to the victim program.

A \emph{write primitive}\index{write primitive} gives the attacker some
level of write access to the victim's memory space. The value written
and the address written to may be controlled by the attacker to various
degrees. The primitive, for example, may allow:

\begin{itemize}
\tightlist
\item
  writing a fixed value to an attacker-controlled address, or
\item
  writing to an address consisting of a fixed base and an
  attacker-controlled offset limited to a specific range (e.g.~a 32-bit
  offset)\todo{Consider
  describing in more detail why the range limitation matters}, or
\item
  writing to an attacker-controlled base address with a fixed offset.
\end{itemize}

Primitives can be further classified according to more detailed
properties. See slide 11 of (Miller,
\protect\hyperlink{ref-Miller2012}{n.d.}) for an example.

The most powerful version of a write primitive is an \emph{arbitrary
write} primitive, where both the address and the value are fully
controlled by the attacker.

A \emph{read primitive}\index{read primitive}, respectively, gives the
attacker read access to the victim's memory space. The address of the
memory location accessed will be controlled by the attacker to some
degree, as for the write primitive. A particularly useful primitive is
an \emph{arbitrary read} primitive, in which the address is fully
controlled by the attacker.

The effects of a write primitive are perhaps easier to understand, as it
has obvious side-effects: a value is written to the victim program's
memory. But how can an attacker observe the result of a read primitive?

This depends on whether the attack is interactive or non-interactive (Hu
et al. \protect\hyperlink{ref-Hu2016}{2016}).

\begin{itemize}
\tightlist
\item
  In an \emph{interactive attack}\index{interactive attack}, the
  attacker gives malicious input to the victim program. The malicious
  input causes the victim program to perform the read the attacker
  instructed it to, and to output the results of that read. This output
  could be any kind of output, for example a network packet that the
  victim transmits. The attacker can observe the result of the read
  primitive by looking at this output, for example parsing this network
  packet. This process then repeats: the attacker sends more malicious
  input to the victim, observes the output and prepares the next input.
  You can see an example of this type of attack in (Beer
  \protect\hyperlink{ref-Beer2020}{2020}), which describes a zero-click
  radio proximity exploit.
\item
  In a \emph{non-interactive (one-shot)
  attack}\index{non-interactive (one-shot)
  attack}, the attacker provides all malicious input to the victim
  program at once. The malicious input triggers multiple primitives one
  after the other, and the primitives are able to observe the effects of
  the preceding operations through the victim program's state. The input
  could be, for example, in the form of a JavaScript program (Groß
  \protect\hyperlink{ref-Grouxdf2020}{2020}), or a PDF file pretending
  to be a GIF (Beer and Groß \protect\hyperlink{ref-Beer2021}{2021}).
\end{itemize}

\todo{The references in this section describe complicated modern exploits.
Consider linking to simpler exploits, as well as some tutorial-level material.}

How does an attacker obtain these kinds of primitives in the first
place? The details vary, and in some cases it takes a combination of
many techniques, some of which are out of scope for this book. But we
will be describing a few of them in this chapter. For example a stack
buffer overflow results in a (restricted) write primitive when the input
size exceeds what the program expected.

As part of an attack, the attacker will want to execute each primitive
more than once, since a single read or write operation will rarely be
enough to achieve their end goal (more on this later). How can
primitives be combined to perform multiple reads/writes?

In the case of an interactive attack, preparing and sending input to the
victim program and parsing the output of the victim program are usually
done in an external program that drives the exploit. The attacker is
free to use a programming language of their choice, as long as they can
interact with the victim program in it. Let's assume, for example, an
exploit program in C, communicating with the victim program over TCP. In
this case, the primitives are abstracted into C functions, which prepare
and send packets to the victim, and parse the victim's responses. Using
the primitives is then as simple as calling these functions. These calls
can be easily combined with arbitrary computations, all written in C, to
form the exploit.

For this cycle of repeated input/output interactions to work, the state
of the victim program must not be lost between the different iterations
of providing input and observing output. In other words, the victim
process must not be restarted.

It's interesting to note that while the read/write primitives consist of
carefully constructed inputs to the victim program, the attacker can
view these inputs as \emph{instructions} to the victim program. The
victim program effectively implements an interpreter unintentionally,
and the attacker can send instructions to this interpreter. This is
explored further in (Dullien \protect\hyperlink{ref-Dullien2020}{2020}).

In the case of a non-interactive attack, all computation happens within
the victim program. The duality of input data and code is even more
obvious in this case, as the malicious input to the victim can be viewed
as the exploit code. There are cases for which the input is obviously
interpreted as code by the victim application as well, as in the case of
a JavaScript program given as input to a JavaScript engine. In this
case, the read/write primitives would be written as JavaScript
functions, which when called have the unintended side-effect of
accessing arbitrary memory that a JavaScript program is not supposed to
have access to. The primitives can be chained together with arbitrary
computations, also expressed in JavaScript.

There are, however, cases where the correspondence between data and code
isn't as obvious. For example, in (Beer and Groß
\protect\hyperlink{ref-Beer2021}{2021}), the malicious input consists of
a PDF file, masquerading as a GIF. Due to an integer overflow bug in the
PDF decoder, the malicious input leads to an unbounded buffer access,
therefore to an arbitrary read/write primitive. In the case of
JavaScript engine exploitation, the attacker would normally be able to
use JavaScript operations and perform arbitrary computations, making
exploitation more straightforward. In this case, there are no scripting
capabilities officially supported. The attackers, however, take
advantage of the compression format intricacies to implement a small
computer architecture, in thousands of simple commands to the decoder.
In this way, they effectively \emph{introduce} scripting capabilities
and are able to express their exploit as a program to this architecture.

So far, we have described read/write primitives. We have also discussed
how an attacker might perform arbitrary computations:

\begin{itemize}
\tightlist
\item
  in an external program in the case of interactive attacks, or
\item
  by using scripting capabilities (whether originally supported or
  introduced by the attacker) in non-interactive attacks.
\end{itemize}

Assuming an attacker has gained these capabilities, how can they use
them to achieve their goals?

The ultimate goal of an attacker may vary: it may be, among other
things, getting access to a system, leaking sensitive information or
bringing down a service. Frequently, a first step towards these wider
goals is arbitrary code execution\index{arbitrary code execution} within
the victim process. We have already mentioned that the attacker will
typically have arbitrary computation capabilities at this point, but
arbitrary code execution also involves things like calling arbitrary
library functions and performing system calls.

Some examples of how the attacker may use the obtained primitives:

\begin{itemize}
\tightlist
\item
  Leak information, such as pointers to specific data structures or
  code, or the stack pointer.
\item
  Overwrite the stack contents, e.g.~to perform a
  \protect\hyperlink{code-reuse-attacks}{ROP attack}.
\item
  Overwrite non-control data, e.g.~authorization state. Sometimes this
  step is sufficient to achieve the attacker's goal, bypassing the need
  for arbitrary code execution.
\end{itemize}

Once arbitrary code execution is achieved, the attacker may need to
exploit additional vulnerabilities in order to escape a process sandbox,
escalate privilege, etc. Such vulnerability chaining is common, but for
the purposes of this chapter we will focus on:

\begin{itemize}
\tightlist
\item
  Preventing memory vulnerabilities in the first place, thus stopping
  the attacker from obtaining powerful read/write primitives.
\item
  Mitigating the effects of read/write primitives, e.g.~with mechanisms
  to maintain \protect\hyperlink{code-reuse-attacks}{Control-Flow
  Integrity (CFI)}.
\end{itemize}

\hypertarget{stack-buffer-overflows}{%
\section{Stack buffer overflows}\label{stack-buffer-overflows}}

A buffer overflow occurs when a read from or write to a data
buffer\href{https://en.wikipedia.org/wiki/Data_buffer} exceeds its boundaries.
This typically results in adjacent data structures being accessed, which has
the potential of leaking or compromising the integrity of this adjacent data.

\hypertarget{refs}{}
\leavevmode\hypertarget{ref-AlephOne1996}{}%
Aleph One. 1996. ``Smashing the Stack for Fun and Profit.'' 1996.
\url{http://www.phrack.org/issues/49/14.html\#article}.

\leavevmode\hypertarget{ref-Beer2020}{}%
Beer, Ian. 2020. ``An iOS Zero-Click Radio Proximity Exploit Odyssey.''
2020.
\url{https://googleprojectzero.blogspot.com/2020/12/an-ios-zero-click-radio-proximity.html}.

\leavevmode\hypertarget{ref-Beer2021}{}%
Beer, Ian, and Samuel Groß. 2021. ``A Deep Dive into an Nso Zero-Click
iMessage Exploit: Remote Code Execution.'' 2021.
\url{https://googleprojectzero.blogspot.com/2021/12/a-deep-dive-into-nso-zero-click.html}.

\leavevmode\hypertarget{ref-Dullien2020}{}%
Dullien, Thomas. 2020. ``Weird Machines, Exploitability, and Provable
Unexploitability.'' \emph{IEEE Transactions on Emerging Topics in
Computing} 8 (2): 391--403.
\url{https://doi.org/10.1109/TETC.2017.2785300}.

\leavevmode\hypertarget{ref-Grouxdf2020}{}%
Groß, Samuel. 2020. ``JITSploitation I: A Jit Bug.'' 2020.
\url{https://googleprojectzero.blogspot.com/2020/09/jitsploitation-one.html}.

\leavevmode\hypertarget{ref-Hicks2014}{}%
Hicks, Michael. 2014. ``What Is Memory Safety?'' 2014.
\url{http://www.pl-enthusiast.net/2014/07/21/memory-safety/}.

\leavevmode\hypertarget{ref-Hu2016}{}%
Hu, Hong, Shweta Shinde, Sendroiu Adrian, Zheng Leong Chua, Prateek
Saxena, and Zhenkai Liang. 2016. ``Data-Oriented Programming: On the
Expressiveness of Non-Control Data Attacks.'' In \emph{2016 Ieee
Symposium on Security and Privacy (Sp)}, 969--86.
\url{https://doi.org/10.1109/SP.2016.62}.

\leavevmode\hypertarget{ref-Miller2012}{}%
Miller, Matt. n.d. ``Modeling the Exploitation and Mitigation of Memory
Safety Vulnerabilities.''
\href{https://2012.ruxconbreakpoint.com/}{Breakpoint 2012}.
\url{https://github.com/Microsoft/MSRC-Security-Research/blob/master/presentations/2012_10_Breakpoint/BreakPoint2012_Miller_Modeling_the_exploitation_and_mitigation_of_memory_safety_vulnerabilities.pdf}.

\leavevmode\hypertarget{ref-Shacham2007}{}%
Shacham, Hovav. 2007. ``The Geometry of Innocent Flesh on the Bone:
Return-into-Libc Without Function Calls (on the X86).'' In
\emph{Proceedings of the 14th Acm Conference on Computer and
Communications Security}, 552--61. CCS '07. New York, NY, USA:
Association for Computing Machinery.
\url{https://doi.org/10.1145/1315245.1315313}.

\leavevmode\hypertarget{ref-Solar1997}{}%
Solar Designer. 1997. ``Getting Around Non-Executable Stack (and Fix).''
1997. \url{https://seclists.org/bugtraq/1997/Aug/63}.

\end{document}
